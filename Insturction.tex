% !TEX root = Insturction.tex
\documentclass[14pt,a4paper,oneside]{extarticle}
\usepackage[utf8]{inputenc} % указывает кодировку документа
\usepackage[T2A]{fontenc} % указывает внутреннюю кодировку TeX 
\usepackage[russian,english]{babel} % указывает язык документа
\usepackage{amssymb,amsfonts,amsmath,cite,enumerate,float,indentfirst,pslatex,palatino,avant,color}
\usepackage{graphicx} 
\usepackage{hyperref}   % для работы ссылок
\usepackage{lmodern}
\usepackage{geometry} % Меняем поля страницы
\usepackage{etoolbox}
\usepackage{tabto}
\usepackage[figurename=Рисунок]{caption}

% блок ниже для точек в содержании (между названием секции и цифрой)
\makeatletter
\patchcmd\l@section{%
  \nobreak\hfil\nobreak
}{%
  \nobreak
  \leaders\hbox{%
    $\m@th \mkern \@dotsep mu\hbox{.}\mkern \@dotsep mu$%
  }%
  \hfill
  \nobreak
}{}{\errmessage{\noexpand\l@section could not be patched}}
\makeatother

% блок ниже для работы ссылок
\hypersetup{
    colorlinks,
    citecolor=black,
    filecolor=black,
    linkcolor=black,
    urlcolor=black
}

\geometry{left=2cm}% левое поле
\geometry{right=1.5cm}% правое поле
\geometry{top=1cm}% верхнее поле
\geometry{bottom=2cm}% нижнее поле


% блок ниже для удобного вывода картинки в формате: \imgh{длина картинки (150mm)}{название картинки (test.png)}{подпись к картинке (тестовая картинка)}
\newcommand{\imgh}[3]
{
    \begin{figure}[h]
    \center{\includegraphics[width=#1]{#2}}
    \caption{#3}
    \label{ris:#2}
    \end{figure}
}


\renewcommand{\rmdefault}{cmr} % Шрифт с засечками
\renewcommand{\sfdefault}{cmss} % Шрифт без засечек
\renewcommand{\ttdefault}{cmtt} % Моноширинный шрифт
\renewcommand{\theenumi}{\arabic{enumi}}% Меняем везде перечисления на цифра.цифра
\renewcommand{\labelenumi}{\arabic{enumi}}% Меняем везде перечисления на цифра.цифра
\renewcommand{\theenumii}{.\arabic{enumii}}% Меняем везде перечисления на цифра.цифра
\renewcommand{\labelenumii}{\arabic{enumi}.\arabic{enumii}.}% Меняем везде перечисления на цифра.цифра
\renewcommand{\theenumiii}{.\arabic{enumiii}}% Меняем везде перечисления на цифра.цифра
\renewcommand{\labelenumiii}{\arabic{enumi}.\arabic{enumii}.\arabic{enumiii}.}% Меняем везде перечисления на цифра.цифра


\title{Инструкция}
\author{Куркурин Никита}
\date{ }

% блок ниже выводит точку после цифры в названии глав и секций
\renewcommand{\thesection}{\arabic{section}.}
\renewcommand{\thesubsection}{\arabic{section}.\arabic{subsection}.}
\renewcommand{\thesubsubsection}{\arabic{section}.\arabic{subsection}.\arabic{subsubsection}}

\usepackage{hyphenat}
\hyphenation{ма-те-ма-ти-ка вос-ста-нав-ли-вать}

\begin{document}


    \maketitle

    \renewcommand{\contentsname}{\center{Содержание}}\tableofcontents % это оглавление, которое генерируется автоматически

    \newpage

    \section{Подготовка сервера}

        \subsection{Работа с пользователем}
            \textbf{adduser sammy \&\& usermod -aG sudo sammy}\\
            создание пользователя и наделение его правами администратора

        \subsection{Фаервол}
            \textbf{ufw app list \&\& ufw allow OpenSSH \&\& ufw enable \&\& ufw status}\\
            Установка фаерловал и разрешение порта ssh, также, проверка его работоспо-\\собности\\

        \LARGE Перезагрузить сервер и зайти под sammy

        \normalsize
        \subsection{Установка необходимого софта}
            \textbf{sudo apt update \&\& sudo apt upgrade \&\& sudo apt install vim tmux -y} \\
            Установка vim'a и tmux'a;
        
        \subsection{Установка Python}
            \textbf{sudo apt install python3-pip python3-dev libpq-dev python3-virtualenv -y } \\
        
        \subsection{Установка MySQL, Nginx}
            \textbf{sudo apt install nginx curl mysql-server -y}
    
    \newpage

    \section{Настройка MySQL}

        \subsection{Первоначальная настройка}

            \noindent\textbf{sudo mysql\_secure\_installation}\\
            Далее такая последовательность ответов:\\

            \begin{itemize}
                \item \textit{If system request about VALIDATE PASSWORD} - N\\
                \item \textit{Password} - root\\
                \item \textit{Remove anon} - Y\\
                \item \textit{Dissallow} - \textbf{N} !!!\\
                \item \textit{Remove test db} - Y\\
                \item \textit{Reload priv.} - Y\\
            \end{itemize}

        \subsection{Работа с пользователем}

            \subsubsection{Настройка рута}
                \begin{itemize}
                    \item \textbf{sudo mysql}
                    \item \textbf{ALTER USER 'root'@'localhost' IDENTIFIED WITH mysql\_native\_password BY 'ПАРОЛЬ;'}
                    \item \textbf{FLUSH PRIVILEGES;}
                    \item \textbf{exit;}
                \end{itemize}

            \subsection{Создаем нового пользователя}
                \begin{itemize}
                    \item \textbf{mysql -u root -p}
                    \item \textbf{пароль который введен был в настройке}
                    \item \textbf{CREATE USER 'sammy'@'localhost' IDENTIFIED BY 'ПАРОЛЬ ДЛЯ SAMMY';}
                    \item \textbf{GRANT ALL PRIVILEGES ON *.* TO 'sammy'@'localhost' WITH GRANT OPTION;}
                    \item \textbf{exit;}
                    \item \textbf{systemctl status mysql.service}
                \end{itemize}
    
    \newpage

    \section{Django}

        \subsection{Запуск на без сервера}
            \begin{itemize}
                \item \textbf{Allow port (if not root != 8000)}
                \item \textbf{Write in ALLOWED\_HOST ip-server.}
                \item \textbf{sudo ufw allow 8001}
                \item start -- \textbf{python manage.py runserver 0.0.0.0:8001}
                \item start with gunicorn -- \textbf{gunicorn --bind 0.0.0.0:8001 web\_kazan.wsgi}
            \end{itemize}
        
        \subsection{Gunicorn}
            \begin{itemize}
                \item \textbf{sudo vim /etc/systemd/system/gunicorn.socket} \\ 
                В этом файле пишем:\\\\
                \texttt{\lbrack Unit\rbrack\\
                Description=gunicorn socket\\
                \\
                \lbrack Socket\rbrack\\
                ListenStream=/run/gunicorn.sock\\
                \\
                \lbrack Install\rbrack\\
                WantedBy=sockets.target\\}

                \item \textbf{sudo vim /etc/systemd/system/gunicorn.service} \\
                В этом файле пишем:\\\\
                \texttt{\lbrack Unit\rbrack\\
                Description=gunicorn daemon\\
                Requires=gunicorn.socket\\
                After=network.target\\
                \\
                \lbrack Service\rbrack\\
                User=sammy
                Group=www-data \\
                WorkingDirectory=/home/sammy/myprojectdir \\
                ExecStart=/home/sammy/myprojectdir/myprojectenv/bin/gunicorn \textbackslash \\
                        --access-logfile - \textbackslash \\
                        --workers 3 \textbackslash \\
                        --bind unix:/run/gunicorn.sock \textbackslash \\
                        myproject.wsgi:application \\
                \\
                \lbrack Install\rbrack \\
                WantedBy=multi-user.target\\}

                \item \textbf{sudo journalctl -u gunicorn.socket}
            \end{itemize}
        
    \newpage
    \section{Nginx}
        \begin{itemize}
            \item \textbf{sudo vim /etc/nginx/sites-available/myproject} \\

            \texttt{server \{ \\
                    listen 8001;\\
                    server\_name SERVER\_IP\_ADDRESS;\\
                    \\
                    location /static/ \{ \\
                        root /home/sammy/web\_kazan; \\
                    \} \\
                    \\
                    location / \{ \\
                        include proxy\_params; \\
                        proxy\_pass http://unix:/run/gunicorn.sock; \\
                    \} \\
                \} \\}
            
            \item \textbf{sudo ln -s /etc/nginx/sites-available/myproject /etc/nginx/sites-enabled}
            \item \textbf{sudo nginx -t}
        \end{itemize}

    \newpage
    \section{Генерация SSL сертификата}

        \begin{itemize}
            \item \textbf{openssl genrsa -out webhook\_pkey.pem 2048}
            \item \textbf{openssl req -new -x509 -days 3650 -key webhook\_pkey.pem -out webhook\_cert.pem}
            \item RU
            \item Moscow
            \item Moscow
            \item .
            \item .
            \item IP\_ADDRESS
            \item .
        \end{itemize}
    
    \newpage
    \section{Supervisor}

        \subsection{Настройка}
            \begin{itemize}
                \item \textbf{sudo apt-get install supervisor}
                \item \textbf{sudo supervisord}
                \item \textbf{sudo service supervisor status}
                \item \textbf{sudo service supervisor start}
                \item \textbf{sudo vim /etc/supervisor/conf.d/APPLICATION\_TITLE.conf}\\

                \texttt{
                    [program:APPLICATION\_TITLE]\\
                    command=/home/sammy/parser\_details/venv/bin/python -u main.py\\
                    directory=/home/sammy/parser\_details\\
                    stdout\_logfile=/home/sammy/parser\_details/logs.txt\\
                    redirect\_stderr=true\\
                    user=sammy\\
                    autostart=true\\
                    autorestart=true\\
                }

                \item \textbf{sudo supervisorctl} - заходим в супервизор
                \item 
                    \begin{itemize}
                        \item \textbf{supervisor > reread} - обновление конфигов
                        \item \textbf{supervisor > add APPLICATION\_TITLE} - добавление конфига в супервизор
                        \item \textbf{supervisor > status} - просмотр статуса всех активных конфигов
                    \end{itemize} 
            \end{itemize}

        \subsection{Доп. параметры}
            \begin{itemize}
                \item \textbf{reread : Reloads conf files}
                \item \textbf{add <program> : Adds a newly created conf file and starts the process}
                \item \textbf{status : Checks all status of programs managed by supervisor}
                \item \textbf{start <program> : Starts the program}
                \item \textbf{restart <program> : Restart the program}
                \item \textbf{tail -f <program> : Watch log file}
                \item \textbf{exit : Exits supervisorctl}
                \item \textbf{help : List commands}
            \end{itemize}

        \subsection{Подключение сайта}
        
            \begin{itemize}
                \item \textbf{sudo vim /etc/supervisor/supervisord.conf} \\
                \texttt{ 
                    \lbrack inet\_http\_server\rbrack\\
                    port=0.0.0.0:9001\\
                    username=1\\
                    password=1
                }
                \item \textbf{sudo ufw allow 9001}
                \item \textbf{sudo service supervisor restart}
            \end{itemize}

\end{document}